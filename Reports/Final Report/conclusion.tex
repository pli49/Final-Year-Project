% ARYAN SECTION

It has been determined that environmental factors have a significant impact on transmission of audible sound. The environmental factors have a larger effect on preformance than limitations within the components used by Limpets. The best general frequency to use is approximately 1.8kHz assuming the exact environment characteristics are not known. Minimal testing has been done, as such a profile of different office environments should be done to confirm these findings, and find more specific information previous studies have not covered.\\

% LAURENCE SECTION

The implementation to generate audio tones is an effective and flexible solution. Due to the implementation the solution is capable of creating multiple different waveforms, e.g. single tone, linear FM sweep, etc.. This flexibility of the waveform generator allowed for quick and effective testing of other aspects of this project. Future development of this waveform generator is needed in to expand upon its capabilities if testing of more varied waveforms are needed.\\

% BILL SECTION

An omni-directional MEMs microphone was decided upon to be used in testing. This microphone during testing was capable of receiving signals from a speaker at distances of 25 meters. The overall performance of this microphone is acceptable, but further testing and profiling needs to be done on the microphone to improve future implementation using the microphone.\\

% TOBY SECTION

The "audio tone start detection" final implementation was to use convolution with a linear FM sweep "chirp". This solution produced minimal error, this error being $\pm 0.1ms$. The implementation has yet to be done on a platform that more accurately measures time, and as such the error in measurements can be decreased. A solution to achieving distance measurements asynchronously was constructed using "virtual walls". Due to the theoretical nature of this solution, the actual performance of the solution will need to be tested. An approach to synchronizing many devices was constructed, though the solution was never tested or implemented. This means the solution requires further development to prove that it will be a suitable solution.\\

The final output of this project is the 3D Mesh of all devices within a network. The solution developed to achieve this is capable of achieving the mesh generation with significant accuracy. There is room for future development that can make this mesh generation capable of handling missing distance measurements and accounting for compounding errors that occur across a network.

% OVERALL
% Suitability

% Quality of data

% Performance of final

% Future directions